\documentclass{rapportCS}
\usepackage{lipsum}
\title{Rapport CentraleSupelec - Template} %Titre du fichier

\begin{document}

%----------- Informations du rapport ---------

\logoentreprise{logos/logoGraphene.png}

\titre{Rapport du stage de fin d'étude} % Titre du fichier

\mention{Mention Sciences des données et de l'information} % Nom de la Mention
\trigrammemention{SDI} % Pour le bas de la page
\master{Master MSc IA} % Nom du master
\filiere{Filière Management de projet et Transformation} % Nom de la filière

\eleve{Jin Ding}

\dates{date début - date fin}

% Informations tuteurs écoles
\tuteurecole{
    Mention : \textsc{Prénom Nom} \\
    prenom.nom@centralesupelec.fr \\
    Filière : \textsc{Prénom Nom} \\ 
    prénom.nom@gmail.com 
} 

\tuteurentreprise{
    \textsc{Prénom Nom} \\
    prénom.nom@entreprise.com 
}

%----------- Initialisation -------------------
        
\fairemarges %Afficher les marges
\fairepagedegarde %Créer la page de garde

%----------- Abstract -------------------
\vspace*{\stretch{1}}
\begin{center}
	\begin{abstract}
        \lipsum[1-2]
    \end{abstract}
\end{center}
\vspace*{\stretch{1}}
\newpage

%------------ Table des matières ----------------

\tabledematieres % Créer la table de matières

%------------ Corps du rapport ----------------


%------------ Introduction ----------------

\section{Introduction} 
\lipsum[3-4] % Effacer cette ligne et écrire le texte souhaité




\newpage
%------------ Related work ----------------

\section{Travaux connexes}
\lipsum[1-2] 





\newpage
%------------ Methodology ----------------

\section{Méthodologie}
\lipsum[2-3]





\newpage
%------------ Application à l'analyse de texte ----------------

\section{Section 1}



%------------- Commandes utiles ----------------

\section{Quelques commandes}

Voici quelques commandes utiles :

%------ Pour insérer et citer une image centralisée -----

\insererfigure{logos/logoCS.png}{3cm}{Légende de la figure}{Label de la figure}
% Le premier argument est le chemin pour la photo
% Le deuxième est la hauteur de la photo
% Le troisième la légende
% Le quatrième le label

Ici, je cite l'image \ref{fig: Label de la figure}


%------- Pour insérer et citer une équation --------------

\begin{equation} \label{eq: exemple}
\rho + \Delta = 42
\end{equation}

L'équation \ref{eq: exemple} est cité ici. 



\end{document}
