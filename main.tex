\documentclass{rapportCS}
\usepackage{lipsum}
\title{Control Theory Report} %Titre du fichier

\begin{document}

%----------- Informations du rapport ---------

\logoentreprise{logos/parissaclay.png}

\titre{Speed Control of an Electrical Drive} % Titre du fichier
\soustitre{Vitesse - Track A - Trinome 32}

\eleve{Jin DING \\
Marcos Ichiro SASAKI \\
Carlos BENÍTEZ ROSETY}

\infocours{TP Report \\ Control Theory}

\dates{19/09/2025 - 22/10/2025}

%----------- Initialisation -------------------
        
\fairemarges %Afficher les marges
\fairepagedegarde %Créer la page de garde

%----------- Abstract -------------------
\vspace*{\stretch{1}}
\begin{center}
	\begin{abstract}
        \lipsum[1-2]
    \end{abstract}
\end{center}
\vspace*{\stretch{1}}
\newpage

%------------ Table des matières ----------------

\tabledematieres % Créer la table de matières

%------------ Corps du rapport ----------------


%------------ Section 1 ----------------

\section{Identification of the System Parameters}
\lipsum[1-3]





\newpage
%------------ Section 2 ----------------

\section{Controller Design and Validation by Simulations}
\lipsum[4-6]





\newpage
%------------ Section 3 ----------------

\section{Experimental Validation of the Controllers}
\lipsum[7-9]





\newpage
%------------ Conclusion ----------------

\section{Conclusion}
\lipsum[10-11]





\newpage
%------------ Application à l'analyse de texte ----------------

\section{Section 1}



%------------- Commandes utiles ----------------

\section{Quelques commandes}

Voici quelques commandes utiles :

%------ Pour insérer et citer une image centralisée -----

\insererfigure{logos/logoCS.png}{3cm}{Légende de la figure}{Label de la figure}
% Le premier argument est le chemin pour la photo
% Le deuxième est la hauteur de la photo
% Le troisième la légende
% Le quatrième le label

Ici, je cite l'image \ref{fig: Label de la figure}


%------- Pour insérer et citer une équation --------------

\begin{equation} \label{eq: exemple}
\rho + \Delta = 42
\end{equation}

L'équation \ref{eq: exemple} est cité ici. 



\end{document}
