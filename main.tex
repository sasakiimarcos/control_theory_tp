\documentclass{rapportCS}
\usepackage{lipsum}
\title{Control Theory Report} %Titre du fichier

\begin{document}

%----------- Informations du rapport ---------

\logoentreprise{logos/parissaclay.png}

\titre{Speed Control of an Electrical Drive} % Title
\soustitre{Vitesse - Track A - Trinome 32}

\eleve{Jin DING \\
Carlos BENÍTEZ ROSETY \\
Marcos Ichiro SASAKI}

\infocours{TP Report \\ Control Theory}

\dates{19/09/2025 - 22/10/2025}

%----------- Initialisation -------------------
        
\fairemarges %Afficher les marges
\fairepagedegarde %Créer la page de garde

%----------- Abstract -------------------
\vspace*{\stretch{1}}
\begin{center}
	\begin{abstract}
        \lipsum[1-2]
    \end{abstract}
\end{center}
\vspace*{\stretch{1}}
\newpage

%------------ Table des matières ----------------

\tabledematieres % Créer la table de matières

%------------ Corps du rapport ----------------


%------------ Section 1 ----------------

\section{Identification of the System Parameters}

\subsection{Introduction}

The objective of this practical work is to design and implement control strategies for the speed regulation of a direct current (DC) electrical drive system. Such systems play a vital role in modern engineering applications, including transportation, robotics, and industrial automation, where precise and stable control is required to ensure efficiency and reliability.

The study combines theoretical modeling with experimental validation and is structured in three main stages. First, the electrical and mechanical parameters of the DC motor are identified experimentally using a platform composed of two coupled DC machines, sensors, and a DC/DC power converter. Second, controllers are designed and tested in simulation using Matlab/Simulink---an inner current loop designed by pole placement and an outer speed loop developed using a linear quadratic (LQ) approach. Finally, these controllers are implemented on the physical system to assess their real-time performance and compare experimental results with simulations.

The overarching goal is to analyze how different control structures and parameters influence the dynamic behavior of the system, and to demonstrate the practical application of modern control techniques in achieving stable and accurate speed regulation in electromechanical systems.

\subsection{Armature resistance and inductance: $R$ and $L$}


\subsection{Voltage constant and torque constant: $K_e = K_c$}


\subsection{Coulomb friction torque and viscous friction coefficient: $C_s$ and $f$}


\subsection{Moment of inertia of the system: $J$}





\newpage
%------------ Section 2 ----------------

\section{Controller Design and Validation by Simulations}
\lipsum[4-6]





\newpage
%------------ Section 3 ----------------

\section{Experimental Validation of the Controllers}
\lipsum[7-9]





\newpage
%------------ Conclusion ----------------

\section{Conclusion}
\lipsum[10-11]





\newpage
%------------ Additional Examples ----------------

\section{Additional Examples}



%------------- Useful Commands ----------------

\section{Useful Commands}

Here are some useful commands:

%------ To insert and cite a centered image -----

\insererfigure{logos/logoCS.png}{3cm}{Figure caption}{Figure label}
% First argument is the path to the image
% Second is the image height
% Third is the caption
% Fourth is the label

Here, I cite the image \ref{fig: Figure label}


%------- To insert and cite an equation --------------

\begin{equation} \label{eq: example}
\rho + \Delta = 42
\end{equation}

Equation \ref{eq: example} is cited here. 



\end{document}
